\cleardoublepage
\chapter*{Introduction}
\markboth{Introduction}{Introduction}
\addcontentsline{toc}{chapter}{Introduction}

Improved computational power, the rise of Big Data and recent developments in machine learning have created new areas of research in finance. This thesis consists of three machine learning applications to risk management. The first two chapters use neural networks to mitigate default and systemic risk and the last chapter is a financial sentiment analysis using natural language processing.  \\

The first chapter builds on the works from \citet{DSW} and \citet{Duan2012}. Both models use a doubly stochastic argument to derive multi-period default probabilities. In particular, they estimate the intensities of two Poisson processes, one governing default and the other governing other exits. \citet{DSW} generates future random values for the covariates using a VAR process while \citet{Duan2012} relaxes this assumption and use forward intensities. The latter specifies the intensities as a linear function of state variables and uses maximum likelihood to estimate the parameters. The first chapter of this thesis extends existing literature by removing the assumption of linear intensities and uses artificial neural networks to estimate the intensities of the Poisson processes. Neural networks are well-suited in this framework because they allow an easy replication of the linear formulation in \citet{Duan2012}. Increasing the network's width and depth allows for non-linearities and out-of-sample Lorenz Curves show that the neural network's approach outperforms the linear assumption for every horizon. Finally, I show what are the most important predictors of default in the short and longer term. Interdepencies in a network are at the heart of systemic risk. The second chapter - connected to the first one as another application of neural networks to risk management - employs Granger causality to identify interconnections among a set of institutions. We build an information measure known as directed information (DI) capable of capturing causal relationships in both linear and non-linear systems. The output of this approach is a directed graph that visualizes the interconnections among a set of time series. Computing DI has both computational and sample complexity which makes it not suitable for inferring the causal structure of large networks. To overcome this problem, we develop a novel approach based on recurrent neural networks that reduces complexity of evaluating DI in high-dimensional settings. We show that our approach performs well both in a linear and non-linear simulated enviornments, then apply it to infer the causal relationships among US firms from 1990 to 2020. \\

The last chapter uses natural language processing to assess the predictive power of social media on stock returns.We scrape 90 millions messages from Stocktwits - a microblogging platform similar to Twitter but designed for finance professionals. One of the challenge in this context is to create a classifier that understands the vocabulary of the message posted by the users. After preprocessing steps, we use TFIDF vectorization to compute the importance of each word in the message. This transforms the messages from words to a vector of numbers which is now in the same dimension as the vocabulary. Then, we build two adversarial logistic regressions using the TFIDF vectors as features and the user-labels as targets. The first model predicts positive or not positive, the second model predicts negative or not negative. When the models agree, we get our label and when they disagree, we treat the tweet as neutral. This procedure allows us to create an artificial neutral class that absorbs all the tweets that do not convey financial information. Finally, with daily intervals, we aggregate tweets predicted sentiments per ticker to compute daily polarity time-series. We then use the daily volume of messages on a given firm to identify sudden peak of activity, indicating a firm event. Computing cumulative average abnormal return and cumulative abnormal polarity in a 41 days window centered at the identified event, we show that abnormal polarities have significant predictive power on the type of event.



